\documentclass[12pt,a4paper]{article}
\usepackage[utf8]{inputenc}
\usepackage{amsmath}
\usepackage{amsfonts}
\usepackage{amssymb}
\usepackage{graphicx}
\author{Yusuf Brima \\
	\small Research Proposal
}
\title{Embodied dyads teaching tool use}
\begin{document}
\maketitle
\date{}
\section{Objective}
We are currently witnessing a, pragmatic turn towards a paradigm that focuses on understanding cognition as  enactive, as skillful activity that involves ongoing interaction with the external world. Here we investigate this hypothesis taking the example of visual communication during tools use. The project tests the hypothesis that visual sampling of information provided by the teacher and tool respectively is optimally sequenced to allow skilled tool use in the task context.
\section{Hypothesis}
We hypothesize that the instructor and pupil, in turn, fixate handle,  effector, the face of the partner. That is individual fixations of, e.g. the pupil on the face of the instructor serve communication purposes and acknowledging the attention to the instruction. Fixations on handle and effector serve sampling information on the action of the instructor, but also inform the instructor of the level of understanding and familiarity of the procedure.
\section{Previous Literature}
 We are building on our previous work investigating visual attention (König et al. 2016).  We could demonstrate that human participants using tools fixate different parts of tools depending on the context. Specifically, familiar tools are fixated more often at the handle (affordance), whereas unfamiliar tools were fixated more often at the effector. This difference was more pronounced in a task where they had to use the tool, as compared to a task where they had to lift the tool. This demonstrates the context and familiarity dependence of visual attention (Keshava et al. in prep.)
\section{Methodology}
We address this problem with modern VR techniques investigating the interaction of two participants while teaching and demonstrating tool use. This combines flexibility, reproducibility, and validity of experiments. Two participants act in a joint virtual reality. A teacher either instructs or demonstrates the use of a tool. The pupil either performs the use of the tool or follows the demonstration. We measure simultaneously eye, head, and body spatial position and orientation to determine the focus of attention and procedure in the task, as well as pupil dilation as an estimate of cognitive load. To investigate causal relationships, we will manipulate the viewing direction of the partner in VR, instructor or pupil respectively, presented to either participant at critical points in time.

\end{document}