%%%%%%%%%%%%%%%%%%%%%%%%%%%%%%%%%%%%%%%%%%%%%%%%%%%%%%%%%%%%%%%%%%%%%%%%%%%%%%%%
%%%%%%%%%%%%%%%%%%%%%%%%%%%%%%%%%%%%%%%%%%%%%%%%%%%%%%%%%%%%%%%%%%%%%%%%%%%%%%%%
%%% Template for AIMS Rwanda Assignments         %%%              %%%
%%% Author:   AIMS Rwanda tutors                             %%%   ###        %%%
%%% Email: tutors2017-18@aims.ac.rw                               %%%   ###        %%%
%%% Copyright: This template was designed to be used for    %%% #######      %%%
%%% the assignments at AIMS Rwanda during the academic year %%%   ###        %%%
%%% 2017-2018.                                              %%%   #########  %%%
%%% You are free to alter any part of this document for     %%%   ###   ###  %%%
%%% yourself and for distribution.                          %%%   ###   ###  %%%
%%%                                                         %%%              %%%
%%%%%%%%%%%%%%%%%%%%%%%%%%%%%%%%%%%%%%%%%%%%%%%%%%%%%%%%%%%%%%%%%%%%%%%%%%%%%%%%
%%%%%%%%%%%%%%%%%%%%%%%%%%%%%%%%%%%%%%%%%%%%%%%%%%%%%%%%%%%%%%%%%%%%%%%%%%%%%%%%


%%%%%% Ensure that you do not write the questions before each of the solutions because it is not necessary. %%%%%% 

\documentclass[12pt,a4paper]{article}

%%%%%%%%%%%%%%%%%%%%%%%%% packages %%%%%%%%%%%%%%%%%%%%%%%%
\usepackage{amsmath}
\usepackage{amssymb}
\usepackage{amsthm}
\usepackage{amsfonts}
\usepackage{graphicx}
\usepackage[all]{xy}
\usepackage{tikz}
\usepackage{verbatim}
\usepackage{float}
\usepackage[left=2cm,right=2cm,top=3cm,bottom=2.5cm]{geometry}
\usepackage{hyperref}
\usepackage{caption}
\usepackage{subcaption}
\usepackage{psfrag}
\usepackage{mathrsfs}
\usepackage{actuarialangle}
\usepackage[T1]{fontenc}
%%%%%%%%%%%%%%%%%%%%% students data %%%%%%%%%%%%%%%%%%%%%%%%
\newcommand{\student}{Yusuf Brima}
\newcommand{\course}{Mathematical Finance}
\newcommand{\assignment}{1}

%%%%%%%%%%%%%%%%%%% using theorem style %%%%%%%%%%%%%%%%%%%%
\newtheorem{thm}{Theorem}
\newtheorem{lem}[thm]{Lemma}
\newtheorem{defn}[thm]{Definition}
\newtheorem{exa}[thm]{Example}
\newtheorem{rem}[thm]{Remark}
\newtheorem{coro}[thm]{Corollary}
\newtheorem{quest}{Question}[section]

%%%%%%%%%%%%%%  Shortcut for usual set of numbers  %%%%%%%%%%%

\newcommand{\N}{\mathbb{N}}
\newcommand{\Z}{\mathbb{Z}}
\newcommand{\Q}{\mathbb{Q}}
\newcommand{\R}{\mathbb{R}}
\newcommand{\C}{\mathbb{C}}

%%%%%%%%%%%%%%%%%%%%%%%%%%%%%%%%%%%%%%%%%%%%%%%%%%%%%%%555
\begin{document}

%%%%%%%%%%%%%%%%%%%%%%% title page %%%%%%%%%%%%%%%%%%%%%%%%%%
\thispagestyle{empty}
%\begin{figure}
%    \centering
%    \includegraphics[width=\textwidth]{aims_rwanda.jpg}
%\end{figure}
\begin{center}
\textbf{AFRICAN INSTITUTE FOR MATHEMATICAL SCIENCES \\[0.5cm]
(AIMS RWANDA, KIGALI)}
\vspace{1.0cm}
\end{center}

%%%%%%%%%%%%%%%%%%%%% assignment information %%%%%%%%%%%%%%%%
\noindent
\rule{17cm}{0.2cm}\\[0.3cm]
Name: \student \hfill Assignment Number: \assignment\\[0.1cm]
Course: \course \hfill Date: \today\\
\rule{17cm}{0.05cm}
\vspace{1.0cm}
\section*{Exercise 1}
\begin{enumerate}
\item[(a)]
	$
\delta(t)  = 0.05 + 0.04t, \qquad 0 \leq t \leq \frac{1}{2}
$\\
$ \delta(t)  =  0.07 -  0.04 \left( t  - \frac{1}{2}   \right)^2, \qquad \frac{1}{2} \leq t \leq 1  $\\
$   C =  \pounds 10,000 $\\
$ \text{Withdraw } =  \pounds 10,000 $
\begin{align*}
u \left( 0,  \frac{1}{4}   \right)   &=  e^{\int_0^{ \frac{1}{4} }  ( 0.05 + 0.04t )dt } \quad =  1.0138\\
u \left( \frac{1}{4} , \frac{1}{2}   \right)   &=  e^{\int_{\frac{1}{4}}^{ \frac{1}{2} }  ( 0.05 + 0.04t )dt } \quad =  1.01638\\
u \left( \frac{1}{2}  ,\frac{3}{4}   \right)   &=  e^{\int_{\frac{1}{2}}^{ \frac{3}{4} }  \left( 0.07 -  0.04 \left( t  - \frac{1}{2}   \right)62  \right)dt } \quad =  1.0174\\
u \left( \frac{3}{4}, 1  \right)   &=  e^{\int_{\frac{3}{4}}^1 \left( 0.07 -  0.04 \left( t  - \frac{1}{2}   \right)62  \right)dt } \quad =  1.01617
\end{align*}
The about she earned at the first 3 months is:\\
\begin{align*}
&=  \pounds  10,000 u \left(  0, \frac{1}{4}   \right) - 1000\\
&=  \pounds  10,000  \time  1.0138 -  \pounds  1000\\
&=  \pounds   9138 \times 1.01638 -  \pounds  1000 \\
&=  \pounds   8287.6804
\end{align*}
The about she earned at the first 9 months is:\\
\begin{align*}
&= 8287.6804 u \left( \frac{1}{2} , \frac{3}{4}   \right)  - \pounds  1000\\
&=  8287.6804 \time  1.0138 - \pounds  1000\\
&=  9138 \times 1.0174 -  \pounds  1000 \\
&= \pounds   7431.88608
\end{align*}

The about she earned over the whole year is:\\
\begin{align*}
&= 7431.88608 u \left( \frac{3}{4} , 1   \right)\\
&=   \pounds  7431.88608 \time  1.01617 \\
&=  \pounds 7552.059678
\end{align*}
\item[(b)] Nominal Interest Rate\\
$u \left( t, t+h  \right) =  1 + h i_h(t) , \quad \text{where} i_h(t) =  \frac{ v\left( t, t+h  \right)  -1  }{ h}$\\
$h  =  \frac{1}{2},  \quad  t =  \frac{1}{2}$\\
\begin{align*}
i_{\frac{1}{2}}(\frac{1}{2}) &=  \frac{    u \left( \frac{3}{4}, 1  \right)  -1} {  \frac{1}{2}}\\
&=  \frac{  e^{ \int_{\frac{1}{2}}^1    \left(  0.07 -  0.04   \left(  t  -  \frac{ 1}{2} \right)^2   \right)  } dt   }{   \frac{1}{2} }\\
&=  \frac{ e^{  \left[   0.08 -  \frac{0.04t^3}{3}  -  0.02t^2  \right]^1_{\frac{1}{2}}   -1 }    }{   \frac{1}{2}}\\
&=  \frac{  e^{\frac{1}{75}} -1 }{   \frac{1}{2}}\\
&=  \frac{   1.0134  -1}{\frac{1}{2}}\\
&=  0.069779
\end{align*}
\end{enumerate}

\section*{Exercise 2}
\begin{enumerate}
		\item[(a)]
				\begin{equation}
						u  =  1 + i
						\label{eq:accu}
				\end{equation}
				\begin{equation}
						v  =  \frac{1}{u},
						\label{eq:discount}
				\end{equation}
				\begin{equation}
						uv = 1
						\label{eq:acc_dis}
				\end{equation}
				\begin{enumerate}
						\item[(i)]  
							\begin{align*}
									a_{\angl{n}} &=  \frac{1}{1 + i}  \left(   \frac{ 1-v^n  } {  1 -   \frac{  1}{1+i}}    \right)\\
									       &=  \frac{1}{1 + i}   \left(   \frac{ 1-v^n  } {  \frac{ 1 + i -1 }{1+i}  }    \right)\\
									       &=  \frac{1}{1 + i}  \left(   (1 -  v^n)  \frac{ 1+i}{i}    \right)\\
									       &=  \frac{1+v^n}{i}
							\end{align*}
						\item[(ii)]  
						      \begin{align*}
						      			s_{\angl{n}}  &=  (1+i)^n a_n\\
						      			       &=  \frac{v^n  -1}{i}
						      \end{align*}
						      From \eqref{eq:accu} $ \implies  i  =  u -1$  \\
						      $\therefore$
						      \begin{align*}
						      			a_{\angl{n}}  &=  \left(   \frac{  \frac{1}{v^n}  -1 }{ u -1}  \right)\\
						      			      &=  \left(   \frac{  \frac{ 1-v^n }{v^n}  }{ u -1}  \right)\\
						      			      &=  \left(  \frac{ 1-v^n }{v^n}   \frac{ 1}{u-1}  \right)\\
						      			      &= \left(  \frac{ 1-v^n}{i}   \frac{i}{v^n}  \right) \\
						      			      &=   \left(  \frac{ 1-v^n}{i}   u^n  \right) \\
						      			      &= \left(  1= i \right) ^n a_{\angl{n}} 
						      \end{align*}
						      
						    \item[(iii)]
						    \begin{align*}
						        		\frac{1}{a_{\angl{n}} }  -  \frac{1}{s_{\angl{n}} }  &=  \frac{i}{ 1 - v^n} -  \frac{1}{(1 +i)^n} \frac{i}{1 - v^n}\\
						        		      &= \frac{i}{1 - v^n}  -  \frac{  i}{   u^n (1-v^n) }\\
						        		      &=   \frac{  iu^n  - i}{  (1-v^n)u^n}\\
						        		      &=  \frac{i(u^n -1) }{  u^n -  (uv)^n } \quad \text{from equation \eqref{eq:discount} } uv=1\\
						        		      &=  \frac{  i(u^n -1) }{ (u^n -1)}\\
						        		      &=  i  
 						    \end{align*}
				\end{enumerate}
			\item[(b)] If $a_n  =  8.3064 $ and $ s_n =  14.2068$\\
					\begin{align*}
								i  &=  \frac{1}{a_n} -  \frac{1}{s_n}	\\
								  &=  \frac{1}{8.3064} - \frac{1}{14.2068} \\
								  &=  0.05\\
								  &= 5\text{\%}
					\end{align*}	
			For $n$,  we solve as follows:\\
			\begin{align*}
						s_n  &=  \left(  1 + i \right)^n a_n\\
						 \left(  1 + i \right)^n &=  \frac{s_n}{a_b} \implies  n \ln(1+i)\\
						 n &=  \frac{  \ln(s_n) }{  \ln(1+i)  } =  11
			\end{align*}							 
\end{enumerate}
\section*{Exercise 3}
\begin{enumerate}
		\item[(a)] Hint
		
				   $v^{m+1}  + v^{m+2} + v^{m+3} + \dots  + v^{m+n}   = a_{\angl{m+n}}  - a_{\angl{n}}$
				   
$\pounds 15,00 \implies 15 \text{ years}  + \pounds 100 \text{ and } \pounds 200 $

At interest rate $ i = 4\% \text{ per annum }  =  0.04 $

The expression for the Present Value $PV$ (initial amount of annual payment) is thus:

\begin{align*}
PV =  5,000  -  Xa_{ \angl{15} } - (100 (a_{ \angl{15} }  - a_{ \angl{5} } )) -  (200( a_{ \angl{15} }  - a_{ \angl{10} })) 
\end{align*}

And to solve for the the initial amount of the annual payment, we proceed as follows:

\begin{align*}
 0 =  5,000  -  Xa_{ \angl{15} } - (100 (a_{ \angl{15} }  - a_{ \angl{5} } )) -  (200( a_{ \angl{15} }  - a_{ \angl{10} })) 
\end{align*}
But from equation \eqref{eq:accu}, \\
$u  =  1.04$\\
 $\text{ it follows that }  v  = \frac{1}{u} \text{ from equation }  \eqref{eq:acc_dis}$\\
\begin{align*}
		\implies   a_{\angl{n}}  &=  \frac{1-v^n}{i} \\
		                     &= \frac{1  -  (1.04)^{-n }}{0.04}
\end{align*}
\begin{align*}
		\ a_{\angl{15}}  &=  \frac{1-v^{15}}{i}  \\
		                     &= \frac{1  -  (1.04)^{-15 }}{0.04}\\
		                     &=  11.12
\end{align*}
\begin{align*}
		 a_{\angl{5}}  &=  \frac{1-v^{5}}{i}  \\
		                     &= \frac{1  -  (1.04)^{-5 }}{0.04}\\
		                     &=  4.45
\end{align*}
\begin{align*}
		a_{\angl{10}}  &=  \frac{1-v^{10}}{i}  \\
		                     &= \frac{1  -  (1.04)^{-10 }}{0.04}\\
		                     &=  8.11
\end{align*}
\begin{align*}
		0  &=  5,000 -  11.12x -  (100(11.12 - 4.45)) -  200(11.12 - 8.11))\\
		x  &=  \pounds 335.52
\end{align*}
\item[(b)] To calculate the loan outstanding at the end of the $3^{\text{rd}}$ year and hence calculate the interest paid in
the $4^{\text{th}}$ year, we proceed as follows:\\
We let Outstanding Loan be the variable $OA$,  therefore,  the Outstanding Loan for the $3^{\text{rd}}$ year is:\\
\begin{align*}
	 OA_{3} &=  C(i+1)^n -  (x(i+1)^{n-1} + x(i+1)^{n-2} + x) \quad \text{ where x is  the initial annual payment}\\
	&= 5,000 u^3 - (xu^2 + xu + x)\\
	&=  5,000((1.04)^3  -  335.52((1.04)^2  +  1.04 +1)\\
	&=  \pounds 4576.96
\end{align*}
The interest paid in the $4^{\text{th}}$ year is:
\begin{align*}
	I &=  OA \times i\\
	                   &= 4576.96 \times 0.04\\
	                   &= \pounds 183.0784
\end{align*}
\item[(c)] If at the end of the 7th year the investor requests that the loan be recalculated with level payments. the new amount of loan is thus:\\
First, we calculate the loan at the 7th year as stated below, where $OL_7$ stands for loan at the 7th year.
\begin{align*}
  OL_7  &= 5,000u^7  - (xu^6 + xu^5 + xu^4 + xu^3 + xu^2 + (x +100)u + (x+100))\\
             &= \pounds  3725.6231
\end{align*}
\begin{align*}
   z  &=  y a_{\angl{8}}\\
       &= y \frac{1-v^n}{ i} \\
       &=  \frac{1 - (1+i )^{-n}}{i}\\
  \therefore y  &= \frac{zi}{ 1 - (1 +i)^{-n}  }\\
  					&= \frac{3725.6231 \times 0.04   }{  1 - (1.04)^{-8}  }\\
  				y 	& =  \pounds 553.34 \quad \text{payable equally for the remaining 8 years}
\end{align*}
\end{enumerate}
\section*{Exercise 4}
\begin{enumerate}
		\item[(a)] 
		Given
		\begin{align*}
				150,000 \bar{S_{\angl{2}}}  + 50,000  \bar{S_{\angl{1}}} &= 25,000 \int_0^{T-2}  e^{-\delta t} dt\\
		\end{align*}
Proof
\begin{align*}
		T  &= 2 - \frac{ \ln[ 1 -  2\delta (1+i)  (4 -  3i)  ]   }{  \delta }
\end{align*}
It follows that
\begin{equation}
	150,000u^2 + 50,000u  =  25,000 \int_0^{T-2}  e^{-\delta t} dt 
	\label{eq:proof}
\end{equation}
\begin{align*}
   	150,000u^2 + 50,000u  &  25,000 \int_0^{T-2}  e^{-\delta t} dt  \quad \text{ dividing both sides by 25,000 we get}\\
   	6u^2 + 2u &=  \int_0^{T-2}  e^{-\delta t} dt 
\end{align*}
We solve for 
\begin{align*}
		 &= 6u^2 + 2u \quad \text{from equation \eqref{eq:accu},  where } u =  1 + i\\
		 &= 6(1 +i)^2  + 2(1+1)\\
		 &= 2(1+i)(3(1+i)+1)\\
		 &= 2(1+i_(4+3i)
\end{align*}
\begin{equation}
		6u^2 + 2u = 2(1+i_(4+3i)
		\label{eq:lhs}
\end{equation}
Also for $\int_0^{T-2}  e^{-\delta t} dt $\\
\begin{align*}
    \int_0^{T-2}  e^{-\delta t} dt  &=   \left[   \frac{-e^{-\delta t} }{\delta} \right]_0^{T-2}\\
    				&=  \frac{  -e^{ -\delta(T-2) }  }{ \delta }  + \frac{1}{ \delta}\\
    				&=  \frac{  1- e^{-\delta(T-2) }   }{ \delta}
\end{align*}
\begin{equation}
		\int_0^{T-2}  e^{-\delta t} dt =  \frac{  1- e^{-\delta(T-2) }   }{ \delta}
		\label{eq:rhs}
\end{equation}
So we equate \eqref{eq:lhs}  and \eqref{eq:rhs} as follows:\\
\begin{align*}
			2(1+i_(4+3i) &=  \frac{  1- e^{-\delta(T-2) }   }{ \delta}\\
			1 -  e^{-\delta(T-2) }  &=  2\delta (1 + i)(4 + 3i)\\
			e^{-\delta(T-2) }  &=  1 -  2\delta (1 + i) (4 +3i)\\ 
			\text{We apply natural log (ln) to both sides}\\
			-\delta(T-2) &=  \ln(1 - 2\delta(1 +i) (4 +3i))\\
			T-2 &= \frac{ -\ln \left(  1 -2 \delta(1+i)  (1 + i)  \right)   }{ \delta }\\
			T &= 2 -   \frac{ \ln \left(  1 -2 \delta(1+i)  (1 + i)  \right)   }{ \delta }
\end{align*}
Given $\delta = \ln(1 + i)  =  \ln(1.075)$\\
\begin{align*}
		T  &=  2 -  \frac{ \left(  1 -2 \ln(1.075) (1.075) (4 + 3(0.075))    \right)  }{ \ln(1.075) }\\
		&= 44.44787 \text{ years}\\
		&= 44.448 \text{ years}
\end{align*}
\end{enumerate}
\end{document}