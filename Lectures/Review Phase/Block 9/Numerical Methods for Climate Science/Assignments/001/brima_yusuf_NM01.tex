%%%%%%%%%%%%%%%%%%%%%%%%%%%%%%%%%%%%%%%%%%%%%%%%%%%%%%%%%%%%%%%%%%%%%%%%%%%%%%%%
%%%%%%%%%%%%%%%%%%%%%%%%%%%%%%%%%%%%%%%%%%%%%%%%%%%%%%%%%%%%%%%%%%%%%%%%%%%%%%%%
%%% Template for AIMS Rwanda Assignments         %%%              %%%
%%% Author:   AIMS Rwanda tutors                             %%%   ###        %%%
%%% Email: tutors2017-18@aims.ac.rw                               %%%   ###        %%%
%%% Copyright: This template was designed to be used for    %%% #######      %%%
%%% the assignments at AIMS Rwanda during the academic year %%%   ###        %%%
%%% 2017-2018.                                              %%%   #########  %%%
%%% You are free to alter any part of this document for     %%%   ###   ###  %%%
%%% yourself and for distribution.                          %%%   ###   ###  %%%
%%%                                                         %%%              %%%
%%%%%%%%%%%%%%%%%%%%%%%%%%%%%%%%%%%%%%%%%%%%%%%%%%%%%%%%%%%%%%%%%%%%%%%%%%%%%%%%
%%%%%%%%%%%%%%%%%%%%%%%%%%%%%%%%%%%%%%%%%%%%%%%%%%%%%%%%%%%%%%%%%%%%%%%%%%%%%%%%


%%%%%% Ensure that you do not write the questions before each of the solutions because it is not necessary. %%%%%% 

\documentclass[12pt,a4paper]{article}

%%%%%%%%%%%%%%%%%%%%%%%%% packages %%%%%%%%%%%%%%%%%%%%%%%%
\usepackage{amsmath}
\usepackage{amssymb}
\usepackage{amsthm}
\usepackage{amsfonts}
\usepackage{graphicx}
\usepackage[all]{xy}
\usepackage{tikz}
\usepackage{verbatim}
\usepackage{float}
\usepackage[left=2cm,right=2cm,top=3cm,bottom=2.5cm]{geometry}
\usepackage{hyperref}
\usepackage{caption}
\usepackage{subcaption}
\usepackage{psfrag}
\usepackage{mathrsfs}
\usepackage{actuarialangle}
\usepackage[T1]{fontenc}
\usepackage{float}
%%%%%%%%%%%%%%%%%%%%% students data %%%%%%%%%%%%%%%%%%%%%%%%
\newcommand{\student}{Yusuf Brima}
\newcommand{\course}{Numerical Methods}
\newcommand{\assignment}{1}

%%%%%%%%%%%%%%%%%%% using theorem style %%%%%%%%%%%%%%%%%%%%
\newtheorem{thm}{Theorem}
\newtheorem{lem}[thm]{Lemma}
\newtheorem{defn}[thm]{Definition}
\newtheorem{exa}[thm]{Example}
\newtheorem{rem}[thm]{Remark}
\newtheorem{coro}[thm]{Corollary}
\newtheorem{quest}{Question}[section]

%%%%%%%%%%%%%%  Shortcut for usual set of numbers  %%%%%%%%%%%

\newcommand{\N}{\mathbb{N}}
\newcommand{\Z}{\mathbb{Z}}
\newcommand{\Q}{\mathbb{Q}}
\newcommand{\R}{\mathbb{R}}
\newcommand{\C}{\mathbb{C}}

%%%%%%%%%%%%%%%%%%%%%%%%%%%%%%%%%%%%%%%%%%%%%%%%%%%%%%%555
\begin{document}

%%%%%%%%%%%%%%%%%%%%%%% title page %%%%%%%%%%%%%%%%%%%%%%%%%%
\thispagestyle{empty}
%\begin{figure}
%    \centering
%    \includegraphics[width=\textwidth]{aims_rwanda.jpg}
%\end{figure}
\begin{center}
\textbf{AFRICAN INSTITUTE FOR MATHEMATICAL SCIENCES \\[0.5cm]
(AIMS RWANDA, KIGALI)}
\vspace{1.0cm}
\end{center}

%%%%%%%%%%%%%%%%%%%%% assignment information %%%%%%%%%%%%%%%%
\noindent
\rule{17cm}{0.2cm}\\[0.3cm]
Name: \student \hfill Assignment Number: \assignment\\[0.1cm]
Course: \course \hfill Date: \today\\
\rule{17cm}{0.05cm}
\vspace{1.0cm}
\section*{Exercise 1}
Considering the following model
\begin{align*}
 v^{\prime \prime \prime}  =  v(t) v^\prime(t) - s(t) \left(   v^{\prime \prime} \right)^2 \quad  \forall  t >  0\\
v(0)  = 1, \quad v^\prime(0)  =  0,  \quad v \prime \prime (0)  =  2
\end{align*}
\begin{enumerate}
  \item 
  We let 
\begin{align*}
  y(t)  &=  v ^\prime (t)\\
  y^ \prime (t)  &= x(t)  =  v^{\prime \prime}(t)\\
  \text{ and }  x^\prime (t)  &=   v^{\prime \prime \prime}
\end{align*}
we therefore get the following system 
\begin{align*}
 v ^\prime (t)  &=    y(t)  \\
   y^ \prime (t)  &=   x(t)   \\
   x^\prime (t)   &=   v(t)  y(t)   - s(t)  (x(t)))^2\\
   \text{ with the following initial conditions }\\
  &  v(0)  =  1 , \quad y(0)   =  0, \quad  x(0)   =  2
\end{align*}
thus 
\begin{align*}
Y^\prime(t)  &= \begin{pmatrix}
     v^\prime (t) \\
     y^\prime (t) \\
     x^\prime (t)
\end{pmatrix}\\
\text{ and } \\
F\left(  t, Y\right)  &=  \begin{pmatrix}
		y(t)\\
		x(t)\\
		v(t) y(t)  - s(t) \left(    x(t)  \right)^2
\end{pmatrix} \quad \quad \left( P \right) \\
\text{ and } \\
Y\left( 0  \right)   &=  \begin{pmatrix}
		1\\
		0\\
		2
\end{pmatrix}
\end{align*}
\item
The explicit Mid-point scheme solution to (P)  using $\delta t_n   =  \delta t$ is as follows
\begin{align*}
 \begin{pmatrix}
     v^\prime (t) \\
     y^\prime (t) \\
     x^\prime (t)
\end{pmatrix}   &=  \begin{pmatrix}
  w(t)  \\
  z(t) \\
  v(t) w(t)  -  s(t)  \left(  z(t) \right)^2
\end{pmatrix}\\
Y^\prime (t) &=   \begin{pmatrix}
     v^\prime (t) \\
     y^\prime (t) \\
     x^\prime (t)
\end{pmatrix} \\
Y &=  \begin{pmatrix}
    v\\
    w\\
    z
\end{pmatrix}\\
F(t,v,w,z)  &=  \begin{pmatrix}
w \\
z\\
vw  - s(t) z^2
\end{pmatrix}\\
Y(0)  &=  Y_0 \\
t_{n  + \frac{1}{2}}  &=  t_n  + \frac{1}{2}  \delta t\\
Y_{n  + \frac{1}{2}} &=  \begin{pmatrix}
   v_{n  + \frac{1}{2}}\\
   w_{n  + \frac{1}{2}}\\
   z_{n  + \frac{1}{2}} 
\end{pmatrix} = \begin{pmatrix}
     v_n \\
     w_n \\
     z_n
   \end{pmatrix}    + \frac{\delta t} {2}  \begin{pmatrix}
       w_n \\
       z_n \\
       v_n  w_n  -  s(t_n)  z_n^2
   \end{pmatrix}\\
   y_{n+1}  &=  \begin{pmatrix}
      v_{n+1}\\
      w_{n+1}\\
      z_{n+1}
   \end{pmatrix} +   \begin{pmatrix}
        w_{n  + \frac{1}{2}}\\
        z_{n  + \frac{1}{2}}\\
        v_{n  + \frac{1}{2}} w_{n  + \frac{1}{2}} -  s_{t_{n  + \frac{1}{2}}}  z_{n  + \frac{1}{2}}
   \end{pmatrix}
\end{align*}
\item Taking $t_0  = 0,  \delta t  = \frac{1}{2} ,  s(t)    =  1+ t^2$ to solve for $v_1  \approx v( t_1) $ and $v_2  \approx   v(t_2) $ with $n =  0$.
\begin{align*}
  Y(0)   =  Y_0   = \begin{pmatrix}
     1\\
     0\\
     2
  \end{pmatrix}\\
  t_{\frac{1}{2}}  =  t_0  + \frac{\delta t}{2}   =  \frac{0.5}{2}   =  \frac{1}{4}\\
  \begin{pmatrix}
     v_{\frac{1}{2}}   \\
     w_{\frac{1}{2}}\
     z_{\frac{1}{2}}
  \end{pmatrix} &=  \begin{pmatrix}
    1\\
    0\\
    2
  \end{pmatrix}     + \frac{1}{4}  \begin{pmatrix}
     0\\
     2\\
     -4
  \end{pmatrix}  =  \begin{pmatrix}
    1\\
    \frac{1}{2}\\
    1
  \end{pmatrix}\\
  \begin{pmatrix}
   v_1\\
   w_1  \\
   z_1 
  \end{pmatrix}  &=    \begin{pmatrix}
    1\\
    0\\
    2
  \end{pmatrix}   + \frac{1}{2}    + \begin{pmatrix}
    \frac{1}{2}\\
    1\\
    \frac{1}{2}   -  ( 1 + (\frac{1}{4})^2)
  \end{pmatrix} =  \begin{pmatrix}
  1.25\\
  0.5\\
  1.72
  \end{pmatrix}
\end{align*}
For $ n  =  1  v_2 \approx  v(t_2)  $
\begin{align*}
    t_{ \frac{3}{2}}    &=  t_1 + \frac{\delta  t}{  2}    =  0.5   + \frac{0.5}{2}   =  \frac{3}{4}\\
\begin{pmatrix}
      v_{ \frac{3}{2}}  \\
     w _{ \frac{3}{2}} \\
     z_{ \frac{3}{2}} 
\end{pmatrix}  &=  \begin{pmatrix}
     1.25\\
     0.5\\
     1.72
\end{pmatrix} + \frac{1}{4}  \begin{pmatrix}
     0.5\\
     1.72\\
     (1.25 \times 0.5)   -  ( 1 + 0.5^2  \times 1.72^2) 
\end{pmatrix} =  \begin{pmatrix}
     1.375\\
     0.93\\
     0.95
\end{pmatrix}\\
\begin{pmatrix}
    v_2\\
    w_2\\
    z_2
\end{pmatrix} &=  \begin{pmatrix}
     1.25\\
     0.5\\
     1.72
\end{pmatrix} + \frac{1}{2}  \begin{pmatrix}
1.375\\
0.93\\
(1.375 \times 0.93)  -  (1 + 0.75^2) \times 0.95^2)
\end{pmatrix} =  \begin{pmatrix}
    1.9375\\
    0.965\\
    1.6543
\end{pmatrix}
\end{align*}
\end{enumerate}
\section*{Exercise 2}
Given the conservation of the energy of a thin spherical air layer (as a model below)
\begin{equation}
C \frac{dT}{dt}    =  \frac{  (1 - \alpha ) S_0}{ 4}    -  \epsilon  \sigma   T^4  \quad \text{ where }   \forall t \in  [0,310] \text{ and } T(0)  =  270 , \quad T  =  T(t)
    \label{eq:ebm}
\end{equation}
where the following quantities are used
\begin{align*}
    C  =  85, \alpha =  0.3,  S_0   =  1367,  \epsilon   = 0.6,  \sigma =   5.67 \times 10^{-8} \\
    T  = T(t)  \quad \text{ globally averaged sufrace temperature }
\end{align*}
\begin{enumerate}
\item To find  the equilibrium temperature $T_{eq}$.
\begin{align*}
    \frac{dT}{dt}     =  0\\
    \implies \frac{  (1 - \alpha ) S_0}{ 4}    -  \epsilon  \sigma   T^4   = 0 \\
    \implies T_{eq}  =  \sqrt[4]{    \frac{   (1  -  \alpha )  S_0 }{ 4 \epsilon  \sigma   }   }
\end{align*}
\item To write $T(t) = Teq + \tilde{T }(t)$  near the equilibrium, where $ \tilde{T }(t)$ is a small time-dependent temperature perturbation$ (| \tilde{T } | << T_{eq})$.  And to  prove that
\begin{equation}
C \frac{d\tilde{T}}{dt}    =  \frac{  (1 - \alpha ) S_0}{ 4}    -  \epsilon  \sigma    \left( T_{eq}  +   \tilde{T}    \right)^4
\end{equation}
\begin{proof}
    \begin{align*}
        T(t)  &=  T_{eq}  + \tilde{T} \quad   |\tilde{T} <<  T_{eq}|\\
        \frac{dT}{ dt}   &=  \frac{dT_{eq}}{   dt}   +  \frac{d\tilde{T}}{dt}  \\
        \implies \frac{dT}{ dt}   &=   \frac{d\tilde{T}}{dt}  \\
        C \frac{dT}{dt}    &=  \frac{  (1 - \alpha ) S_0}{ 4}    -  \epsilon  \sigma   T^4 \\
        C \frac{d\tilde{T}}{dt}    &=  \frac{  (1 - \alpha ) S_0}{ 4}    -  \epsilon  \sigma    \left( T_{eq}  +   \tilde{T}    \right)^4
    \end{align*}
\end{proof}
\item Assuming that 
\begin{equation}
      \left(      1 + \frac{\tilde{T}}{T_{eq}}  \right)^4   =   \left(     1  +  \frac{4\tilde{T}}{T_{eq}}   \right)
\end{equation}
\begin{proof}
    \begin{align*}
        \frac{d\tilde{T}}{dt}    &=   - \left(    \frac{  4 \epsilon  \sigma T_{eq}^3   }{C}     \right)  \tilde{T}\\
        C \frac{dT}{dt}   &=    \frac{(1-\alpha) S_0}{4}     -  \epsilon  \sigma   T_{eq}^4      \left(      1 + \frac{\tilde{T}}{T_{eq}}  \right)^4 \\
        &=  \frac{(1-\alpha) S_0}{4}     -  \epsilon  \sigma   T_{eq}^4      \left(      1 + \frac{4\tilde{T}}{T_{eq}}  \right)\\
        &=  \frac{(1-\alpha) S_0}{4}     -  \epsilon  \sigma   T_{eq}^4      \left(     4 \epsilon   \sigma   T_{eq}^3    \right) \tilde{T}\\
        \implies C \frac{d \tilde{T}}{dt}   &=  -  \left(     4 \epsilon   \sigma   T_{eq}^3    \right) \tilde{T}\\
       \implies   \frac{d \tilde{T}}{dt}    &=   -  \left(     \frac{
       4 \epsilon   \sigma   T_{eq}^3}{C}    \right) \tilde{T}
    \end{align*}
\end{proof}
\item Given $\tilde{T}(0) = 10, A  =  10,  t \in [0,310] $, we  find the exact solution of \eqref{eq:ebm} as follows.
\begin{align*}
\frac{d \tilde{T}}{dt}    &=   -  \left(     \frac{
       4 \epsilon   \sigma   T_{eq}^3}{C}    \right) \\
       \int \frac{d \tilde{T}}{dt}    &=   -  \left(     \frac{
       4 \epsilon   \sigma   T_{eq}^3}{C}    \right) \int dt \\
       \ln(\tilde{T})   &=   -  \left(     \frac{
       4 \epsilon   \sigma   T_{eq}^3}{C}    \right) t + A_1\\
       \tilde{T}  &=      A \exp^{  -  \left(     \frac{
       4 \epsilon   \sigma   T_{eq}^3}{C}    \right) t      }\\
       \tilde{T}  &=  10   \exp^{  -  \left(     \frac{
       4 \epsilon   \sigma   T_{eq}^3}{C}    \right) t      }
\end{align*}
\item To solve \eqref{eq:ebm} numerically,  one considering the following scheme
\begin{align*}
(P) \left\{
                \begin{array}{ll}
               			y_0  =  y(t_0)\\
               			K_1 =  f(t_n,y_n)\\
               			t_{n  + \frac{3}{4}}  = y_n  + \frac{3}{4}  \delta t  K_1\\
               			K_2  =   f  (t_{n  + \frac{3}{4}}   ,   y_{n  + \frac{3}{4}})\\
               			y_{n+1}   =   y_n  + \frac{1}{3}  \delta t  (K_1  + 2 K_2)
                \end{array}
              \right.
\end{align*}
\begin{enumerate}
			\item[(a)] Prove that (P) is the one-step method.
				\begin{proof}
				         
				         Which means 
						\begin{align*}
						       y_{n+1}   &=   y_n  + \frac{1}{3}  \delta t  (K_1  + 2 K_2)   \equiv  y_{n+1}   =   y_n  +  \delta t   \phi (t_n , y_n,  \delta t)\\
						       y_{n+1}   &=     y_n  + \frac{1}{3}  \delta t \left(   f(t_n,   y_n ) + 2f( t_{n  + \frac{3}{4}}  , y_{n  + \frac{3}{4}}   )    \right)\\
						       &=   y_n  + \frac{1}{3}     \delta t     \left[    f(t_n  ,  y_n)  +   2f\left(  t_n   +  \frac{3}{4}     \delta  t    ,   y_n   + \frac{3}{4}   \delta t  f(t_n , y_n)    \right)           \right]  \\
						       &=    y_n   + \frac{1}{3}  \delta t   \phi (t_n , y_n,  \delta t) \\
						       &   \text{ where }   \phi (t_n , y_n,  \delta t)  =   \left[    f(t_n  ,  y_n)  +   2f\left(  t_n   +  \frac{3}{4}     \delta  t    ,   y_n   + \frac{3}{4}   \delta t  f(t_n , y_n)    \right)           \right] 
 						\end{align*}
				\end{proof}
			\item[(b)]
\end{enumerate}
\end{enumerate}
\end{document}